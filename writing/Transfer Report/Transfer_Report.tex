\documentclass[12pt]{report}
\usepackage[a4paper, total={6.5in, 8in}]{geometry}
\usepackage{xargs}
\usepackage{amsmath,amssymb}
\usepackage{hyperref}
\usepackage{physics}
\usepackage{graphicx}

\newcommand{\sgn}{\text{sgn}}
\newcommand{\seq}[1]{\langle #1\rangle}
\newcommand{\asc}[1]{\upharpoonleft #1 \upharpoonright}
\newcommand{\hc}{^\dagger}
\newcommand{\inv}{^{-1}}
\newcommand{\Sym}{\text{Sym}}
\newcommand{\normord}[1]{:\mathrel{\mspace{2mu}#1\mspace{2mu}}:}

\begin{document}

	\title{Bosonic and fermionic particle-preserving coherent states, their dynamics and applications\\\hfill\\Transfer Report}
	\author{Michal Horanský}
	\maketitle
	
	\chapter*{Abstract}
	Coherent states-based methods have been previously developed and utilised for particle-preserving bosonic systems, such as the Bose-Hubbard model. This work is divided into two parts: Firstly, a new method is proposed for the bosonic systems, where the basis coherent states are propagated separately to create a frozen dynamical basis, which lowers the time-complexity significantly. Second, an analogous method is proposed for particle-preserving fermionic systems, and particularly the electronic structure of a molecule, for which the goal of this method is to find the ground state. For this, multiple methods are outlined, and relevant properties of the fermionic coherent states are obtained.
	
	
	\tableofcontents
	
	
	\chapter{Background}
	
	To find the dynamics of a given system, one typically needs to solve the Schrodinger equation given by the system's Hamiltonian. For complex systems which can be found in nature, an analytic solution rarely exists. To solve this partial differential equation computationally, one needs to chose a basis for the Hilbert space and, since it typically is infinite or even continuous, sample it to obtain a basis sample onto which the initial state is decomposed and on which it is propagated. This sampling choice is always limiting and must be carefully justified, as omitting elements of the Hilbert space may render the analysis inaccurate.
	
	Coherent states are particular states in the Hilbert space which follow classical trajectories, remain coherent, and minimise position-momentum uncertainty. Choosing them as the basis, a very small sample is typically suitable to capture the physics of a wide variety of systems and starting conditions \cite{grossmann}. This is an example of a semi-classical approximation.
	
	\section{Utility and scope of semi-classical methods}
get that adiabatic shit outta here lil bro
	
\subsection{Bosonic system 1: Bose-Hubbard model}
he bossin
	
\subsection{Bosonic system 2: Displaced harmonic trap}
me when i quantise the modes
	
\subsection{Fermionic system: molecular electronic structure}
ground state would be nice aha

	
	
	
	\section{Coherent states: a hundred year-long history}
The study of classical trajectories present in quantum dynamics dates back to Schrodinger's work in 1926, and remained an active area of study ever since. Multiple generalisations were constructed for the initial concept of coherent states, and the specific properties of coherent states of many particular systems were found and subsequently utilised in semi-classical methods. In this section, I outline a brief history of the theoretical treatment of coherent states leading up to the framework used for the $SU(M)$ bosonic/fermionic coherent states.
	
\subsection{Schrodinger: the harmonic oscillator}
In 1926, Schrodinger formulated so-called "classical states" of the simple harmonic oscillator \cite{harmonic_classical_states}. Consider the Hamiltonian of the simple harmonic oscillator
\begin{equation}
\hat{H}=\omega \left(\hat{a}\hc\hat{a}+\frac{1}{2}\right)
\end{equation}
where $\omega$ is the characteristic scale of the potential. Then consider the following superposition of energy eigenstates $\ket{n}$, characterised by a single complex parameter $\alpha$:
\begin{equation} \label{eq:schrodinger_CS}
\ket{\alpha}=N(\alpha)\sum_{i=0}^\infty \frac{\alpha^n}{\sqrt{n!}}\ket{n}
\end{equation}
where $N(\alpha)$ is a normalisation factor. This state has the following properties:
\begin{enumerate}
	\item The expected values of $\hat{x}$ and $\hat{p}$ are proportional to the real and imaginary components of $\alpha$, respectively.
	\item These expected values evolve according to the "classical" Hamiltonian equation, which can therefore be succintly written as
	\begin{equation}
	\dot{\alpha}=-i\pdv{H(\alpha, \alpha^*)}{\alpha^*}
	\end{equation}
	where $H(\alpha, \alpha^*)$ is the classical simple harmonic oscillator Hamiltonian as given by the expected values of $\hat{x}, \hat{p}$.
	\item The position and momentum uncertainties of $\ket{\alpha}$ are minimal: $\Delta x\Delta p=\frac{1}{2}$.
	\item The state remains "coherent", i.e. as it evolves, the parameter $\alpha(t)$ changes, but the wave-state remains describable by the construction in Eq. \ref{eq:schrodinger_CS}.
\end{enumerate}
The significance of these states is immediately obvious: although fundamentally non-classical, the Schrodinger equation with the simple harmonic oscillator Hamiltonian permits "wave-groups" (as dubbed by Schrodinger) whose expectation values not only follow non-trivial classical trajectories, but which remain compact (both in space and in momentum representation!) and coherent.
	
\subsection{Glauber: field coherent states}
In 1963, Glauber generalised Schrodinger's construction to the Hamiltonian which describes the interaction between an atomic system and an electromagnetic field, forming field coherent states\footnote{Also called Glauber coherent states.} \cite{field_coherent_states}. According to Glauber, the following three properties all lead to the construction of field coherent states \cite[p. 869]{ZFG}:
\begin{enumerate}
	\item The coherent state is an eigenstate of the annihilation operator
	\begin{equation}
	\hat{a}\ket{\alpha}=\alpha\ket{\alpha}
	\end{equation}
	\item The coherent state is obtained by applying a displacement operator on the vacuum state
	\begin{equation}
	\ket{\alpha}=\hat{D}(\alpha)\ket{0}\qq{where} \hat{D}(\alpha)=\exp(\alpha\hat{a}\hc-\alpha^*\hat{a})
	\end{equation}
	\item The coherent state is a state with minimal position-momentum uncertainty.
\end{enumerate}
However, as remarked by Zhang, Feng, and Gilmore in \cite{ZFG}, the third property is not sufficient for a unique construction.

Field coherent states showed that "classical states" are present in systems beyond the simple harmonic oscillator, but Glauber's construction still was not applicable to general Hamiltonians--specifically, it fails when the Hilbert space is finite, as no eigenstate of the annihilation operator exists! This is particularly relevant when studying systems which preserve total particle number, and thus further generalisation was required.
	
\subsection{Perelomov and Gilmore: generalised field coherent states}
Such a generalisation of Glauber's coherent states for arbitrary Hamiltonians was indeed found independently by Perelomov \cite{perelomov_og} and Gilmore \cite{gilmore_og} in 1972. This construction discards the approach via annihilation operator eigenstates, and instead fully generalises the approach through the displacement operator (which, in retrospect, fully recovers Schrodinger's classical states and Glauber's field coherent states). Perelomov and Gilmore's construction follows these steps:
\begin{enumerate}
	\item Firstly, the transition operators $\hat{T}_i$ of the Hilbert space $\mathcal{H}$ are identified. These are operators which, given any state in $\mathcal{H}$, can reach any other state in $\mathcal{H}$ by finitely-many sequential applications to the initial state. These operators must form a Lie algebra. Also, the Hamiltonian has to be expressable as a function of the transition operators (not necessarily a linear function).
	\item Secondly, a reference state $\ket{\phi_0}$ is chosen. This choice is arbitrary.
	\item Thirdly, two Lie groups are identified: the dynamical group $G$, which is the Lie group associated with the Lie algebra formed by the transition operators $\hat{T}_i$, and its stability subgroup $\hat{H}$, which is the group of all elements in $G$ which leave $\ket{\phi_0}$ invariant up to a phase. The quotient group $G/H$ then contains elements $\hat{D(z_j)}=\exp(\sum_j z_j\hat{T_j})$, where the sum over $j$ contains those transition operators which are not in the stability group Lie algebra. Thus we obtain a map which assigns one coherent state to every element of the quotient group:
	\begin{equation}
	\ket{z}=N(z, z^*)\exp(\sum_jz_j\hat{T}_j)\ket{\phi_0}
	\end{equation}
	where $N$ is a normalisation function, which is necessary since $\hat{D}(z)$ is not in general unitary: the parameters $z_j$ are complex and the transition operators are not required to be Hermitian.
\end{enumerate}
These states posses many of the aforementioned properties, most importantly, classical-like trajectories, as will be discussed in Sec. \ref{sec:mathematical_approach}.


	
	
	
	\section{Mathematical approach to CS-based methods}\label{sec:mathematical_approach}
In this section, we firstly discuss the mathematical properties of generalised field coherent states and their ensembles, and then the particular construction of $SU(M)$ coherent states.
	
\subsection{Topology of the CS parameter space}
This subsection is a summary of relevant information as included by Viscondi, Grigolo, and de Aguiar in their review of generalised field coherent states \cite{aguiar}.

When applying the quotient-space displacement operator on the reference state, the resulting states are no longer normalised. We will adopt the notation of Viscondi, Grigolo, and de Aguiar, and denote unnormalised elements of the Hilbert space as bras and kets with curly brackets. Then, the unnormalised coherent state is directly obtained as
\begin{equation}
\unnormket{z}=\hat{D}(z)\ket{\phi_0}
\end{equation}
and a normalised coherent state is constructed by scaling by a normalisation factor:
\begin{equation}
\ket{z}=N(z^*, z)\unnormket{z}\qq{where}N(z_a^*, z_b)=\unnormbraket{z_a}{z_b}^{-\frac{1}{2}}
\end{equation}
The quotient space can be characterised by the following metric:
\begin{equation}
g(z_a^*, z_b)=\pdv{\ln\unnormbraket{z_a^*}{z_b}}{z_b}{z_a^*}
\end{equation} 
Then, the coherent states remain coherent under time evolution, with the parameter $z$ evolving according to a Hamiltonian equation on a curved manifold:
\begin{equation} \label{eq:single_z_prop}
\dot{z}_i=-i\xi^T_{ij}(z^*, z)\pdv{H(z^*, z)}{z_j^*}
\end{equation}
where $H(z^*, z)=\mel{z}{\hat{H}}{z}$ is the effective Hamiltonian and $\xi(z^*, z)$ is the inverse of the quotient-space metric $g(z^*, z)$.

	
\subsection{Fully variational equations of motion}
Here we employ the paradigm discussed in previous sections which encompasses the heart of this kind of semi-classical approaximation: a limited sample of classical trajectories. We will approximate our Hilbert space by a discrete sample of $N$ coherent states, each characterised by its parameters $z_a$, which evolve in time. Onto this basis sample, we decompose an arbitrary wavestate $\ket{\Psi(t=0)}$ whose dynamics we are interested in:
\begin{equation}
\ket{\Psi(t=0)}=\sum_{a=1}^N A_a(t=0)\unnormket{z_a(t=0)}
\end{equation}
Note that I opt to use the unnormalised coherent states as the basis, rather than their normalised counterparts. This does not affect the physicality of the solution, which shall remain normalised, and it serves to simplify the equations of motion by omitting the normalisation factors $N(z_a)$.

Now, we can apply the Schrodinger Lagrangian $\hat{L}=i\dv{t}-\hat{H}$ onto our system of free coordinates $(A_a, z_a)$ to obtain a system of first-order equations of motion, which can be computationally solved to obtain the trajectories of the basis, the decomposition coefficient evolution, and hence the evolution of the initial wavestate. This approach is a slight modification of the approach used by Qiao and Grossmann in \cite{grossmann}.

Let us begin by obtaining the derivatives of unnormalised coherent states:
\begin{align}
\pdv{z_j}\unnormket{z}&=\pdv{z_j}\exp(z_i\hat{T}_i)\ket{\phi_0}=\hat{T}_j\unnormket{z}\\
\dv{t}\unnormket{z}&=\dv{t}\exp(z_i\hat{T}_i)\ket{\phi_0}=\dot{z}_i\hat{T}_i\unnormket{z}
\end{align}
Then the Lagrangian of the trial wavestate is
\begin{align*}
L=\sum_{m,n}^N\left(iA_m^*\dot{A}_n\unnormbraket{z_m}{z_n}+iA_m^*A_n\sum_i \dot{z}_{n,i} \unnormmel{z_m}{\hat{T}_i}{z_n}-\unnormmel{z_m}{\hat{H}}{z_n}\right)
\end{align*}
and the equations of motion obtained by differentiating with respect to $A_m^*, z_{m,i}^*$, respectively, are
\begin{align}
i\sum_{n}^N \left(\dot{A}_n\unnormbraket{z_m}{z_n}+A_n\sum_i \dot{z}_{n,i}\unnormmel{z_m}{\hat{T}_i}{z_n} \right)&=\sum_{n}^N A_n\unnormmel{z_m}{\hat{H}}{z_n}\\
i A_m^*\sum_{n}^N \left(\dot{A}_n \unnormmel{z_m}{\hat{T}_i\hc}{z_n} + A_n \sum_j \dot{z}_{n,j} \unnormmel{z_m}{\hat{T}_i\hc \hat{T}_j}{z_n} \right) &= A_m^*\sum_{n}^N A_n\unnormmel{z_m}{\hat{T}_i\hc\hat{H}}{z_n}
\end{align}
Note that for $N=1$ (i.e. the wavestate is a pure coherent state), these equations are equivalent to Eq. \ref{eq:single_z_prop}.

It is helpful to express this in matrix form, with the coordinate vector defined as
\begin{equation}
\vec{Q}=\mqty(\vec{Q}^A\\\vec{Q}^z) \qquad \vec{Q}^A_m=A_m \qquad \vec{Q}^z=\mqty(\vdots\\\vec{Q}^{z,i}\\\vdots) \qquad \vec{Q}^{z,i}_m=z_{m,i}
\end{equation}
where the first-order matrix equation of motion becomes
\begin{equation}
i\Omega\dv{t}\vec{Q}=\vec{R} \qq{where} \vec{R}=\mqty(\vec{R}^A\\\vec{R}^z) \qquad \vec{R}^z=\mqty(\vdots\\\vec{R}^{z,i}\\\vdots)
\end{equation}
and where
\begin{equation} \label{eq:dynamical_matrix}
\Omega = \mqty(
	\Omega^{AA} & \Omega^{A,1} & \cdots & \Omega^{A,j} & \cdots\\
	\Omega^{1,A} & \Omega^{1,1} & \cdots & \Omega^{1,j} & \cdots\\
	\vdots & \vdots & \ddots & \vdots &\\
	\Omega^{i,A} & \Omega^{i,1} & \cdots & \Omega^{ij} & \cdots\\
	\vdots & \vdots & & \vdots & \ddots
)
\end{equation}
The particular elements of the dynamical matrix $\Omega$ and the weight vector $\vec{R}$ are then identified as
\begin{align}
\Omega^{AA}_{mn} &= \unnormbraket{z_m}{z_n}\\
\Omega^{A,i}_{mn} &= A_n\unnormmel{z_m}{\hat{T}_i}{z_n}\qq{and} \Omega^{i,A}=(\Omega^{A,i})\hc\\
\Omega^{ij}_{mn} &= A_m^*A_n\unnormmel{z_m}{\hat{T}_i\hc \hat{T}_j}{z_n}\\
R^A_m &= \sum_n^N A_n\unnormmel{z_m}{\hat{H}}{z_n}\\
R^{z,i}_m &= A_m^*\sum_n^N A_n \unnormmel{z_m}{\hat{T}_i\hc \hat{H}}{z_n}
\end{align}
Note that $A_m^*$ could have been factored out of the second group of equations, but was deliberately left in to demonstrate that $\Omega$ is Hermitian.

This form of the equation of motion is especially useful, since it yields itself readily to primitive methods of numerical integration (the method of choice in this work is the Dormand-Prince method, see below). To propagate the wavestate by a finite time-difference $\Delta t$, one simply inverts the dynamical matrix $\Omega$ at the current time. Therefore the time complexity of this approach is limited by the evaluation of $R$ and $\Omega$ and the inversion of $\Omega$, both of which are polynomial in the basis size and the number of transition operators.

\subsection{$SU(M)$ coherent states}
Consider a system characterised by $M$ modes and $S$ particles, with both of these numbers remaining constant. The most general transition operator is then
\begin{equation}
\hat{T}_{ij}=\hat{a}_i\hc\hat{a}_j
\end{equation}
where we remain agnostic about whether $\hat{a}$ denotes the bosonic or the fermionic annihilation operator. In either case, applying this operator on an arbitrary occupancy basis state within the configuration space yields either zero or another basis state within the configuration space. It can also be shown by applying the bosonic and fermionic commutation relations that for both of these cases, the transition operators satisfy the following commutation relations:
\begin{equation}
\comm{\hat{T}_{ij}}{\hat{T}_{i'j'}}=\hat{T}_{ij'}\delta_{i'j}-\hat{T}_{i'j}\delta_{ij'}
\end{equation}
Since the commutator is contained in the vector space spanned by $\hat{T}$, the transition operators form a Lie algebra.

\subsubsection{On transformations of the dynamical Lie algebra}
Note that a coherent state with parameter $z$ is specified by the linear combination $z_i\hat{T}_i$. Hence, we can specify a new dynamical Lie algebra $\hat{T}'_i$ formed as an arbitrary complex linear combination of $\hat{T}_i$. These two Lie algebras correspond to the same coherent state space, with the parametric maps related by the same linear transformation.

Firstly, we choose to transform the mode-number operators $\hat{a}_i\hc\hat{a}_i$ like so:
\begin{align}
\hat{S}&=\sum_i^M \hat{a}_i\hc\hat{a}_i\\
\hat{H}_i&=\hat{a}_{i+1}\hc\hat{a}_{i+1}-\hat{a}_i\hc\hat{a}_i,\qquad i=1,2\dots M-1
\end{align}
The operator $\hat{S}$ is uninteresting, because every state in the configuration space is its eigenstate with eigenvalue $S$, and therefore $\hat{S}$ is always in the stability subgroup. We can therefore ignore it.

The transformed Lie algebra constitutes the generators of real traceless $(M\cross M)$ matrices, and therefore the full dynamical group is $U(1)\cross SL(M,\mathbb{R})$. Ignoring the $U(1)$ direct product, as it is always fully contained in the stability subgroup, we may denote the dynamical group simply as $SL(M,\mathbb{R})$.

\subsubsection{On the difference between $SU(M)$ and $SL(M,\mathbb{R})$}
	When transforming the basis $\hat{T}_{ij}$, one may choose to complexify it and form a set of Hermitian and anti-Hermitian operators ($\hat{T}_{ij}+\hat{T}_{ji}$ and $i(\hat{T}_{ij}-\hat{T}_{ji})$), respectively. This complexification is allowed in the sense that the resulting coherent state space is equivalent. Identifying the resulting operators with the generalised Gell-Mann matrices yields the dynamical group $SU(M)$ \cite{gellmann}. Although resulting in a different quotient group $G/H$, the coherent states are identical to our $SL(M,\mathbb{R})$ construction. For the details of this construction, see Sec. 2.2.3 in the paper by Viscondi, Grigolo, and de Aguiar \cite{aguiar}.
	
	For the sake of simplicity and adherence to existing literature, we will refer to this construction as $SU(M)$ coherent states, but we will still use the exponential map with the $SL(M,\mathbb{R})$ generators, as it leads to much simpler mathematical treatment.

\subsection{General approach to particle-preserving CS decomposition} \label{sec:general_decomposition}
	This method is based on the approach in \cite[App. E]{buonsante}. Suppose we have a reference state $\ket{\phi_0}$, and the quotient space of the dynamical group of some system is transversed by the exponential map of the operator $\hat{F}(z)$, so that
	\begin{equation}
	\unnormket{z}=e^{\hat{F}(z)}\ket{\phi_0}
	\end{equation}
	and $\hat{F}(z)$ is a linear combination of transition operators which all destroy the vacuum state\footnote{This is characteristic of dynamical groups which preserve total particle number.}. Define $\hat{\phi}_0$ as such an operator so that
	\begin{equation}
	\hat{\phi}_0\ket{\rm vac.} = \ket{\phi_0}
	\end{equation}
	Then we can write
	\begin{equation}
	\unnormket{z}=e^{\hat{F}(z)}\hat{\phi}_0e^{-\hat{F}(z)}\ket{\rm vac.}
	\end{equation}
	using the fact that $\hat{F}(z)\ket{\rm vac.}=0$. We now express the operator product using Hadamard's lemma:
	\begin{equation}\label{eq:hadamard sum}
	e^{\hat{F}(z)}\hat{\phi}_0e^{-\hat{F}(z)}=\sum_{r=0}^\infty \frac{1}{r!}\comm{\hat{F}(z)}{\hat{\phi}_0}_r
	\end{equation}
	where
	\begin{equation}
	\comm{\hat{A}}{\hat{B}}_r=\comm{\hat{A}}{\comm{\hat{A}}{\hat{B}}_{r-1}}\qq{and}\comm{\hat{A}}{\hat{B}}_0=\hat{B}
	\end{equation}
	are the repeated commutators. The unnormalised coherent state is obtained by acting with this sum on the vacuum state.

	
	
	
	
	
	\chapter{Current work}
	i work hard yes
	
	\section{Bosonic $SU(M)$ coherent states}
	
\subsection{Construction}
In this subsection, I describe the properties of $SU(M)$ coherent states constructed on bosonic annihilation and creation operators.

\subsubsection{Reference state and displacement operator}
A suitable reference state for a bosonic system of $M$ modes and $S$ particles is
\begin{equation}
\ket{\phi_0}=\ket{0,0\dots 0, S}
\end{equation}
Finding the stability subgroup of the dynamical group for this reference state, we ultimately obtain the displacement operator
\begin{equation}
\hat{D}(z)=\exp(\sum_{i=1}^{M-1}z_i\hat{b}\hc_i\hat{b}_M)
\end{equation}
The bosonic $SU(M)$ coherent state is then
\begin{equation}
\ket{z}=N(z)\exp(\sum_{i=1}^{M-1}z_i\hat{b}\hc_i\hat{b}_M)\ket{\phi_0}
\end{equation}
where $z$ is a vector of $M-1$ complex parameters

\subsubsection{Decomposition into the occupancy basis}
As shown by Buonsante and Penna in \cite{buonsante}, this coherent state can be decomposed into the occupancy basis as follows:
\begin{equation} \label{eq:bosonic_decomposition}
\ket{\vec{z}}=\frac{N(z)}{\sqrt{S!}}\left(\hat{a}^\dagger_M+\sum_i^{M-1}z_i\hat{a}^\dagger_i\right)^S\ket{0,0\dots 0}
\end{equation}
This result can be obtained by following the method outlined in Sec. \ref{sec:general_decomposition}. Applying the multinomial theorem gives the explicit superposition of occupancy basis states equivalent to the coherent state:
\begin{equation}
\ket{z}=N(z)\sum_{k_1+k_1+\dots +k_M=S}\sqrt{\frac{S!}{k_1!k_2!\dots k_M!}}z_1^{k_1}z_2^{k_2}\dots z_{M-1}^{k_{M-1}}\ket{k_1,k_2\dots k_M}
\end{equation}

\subsubsection{Overlap and normalisation}
Using Eq. \ref{eq:bosonic_decomposition} and Wick's theorem, the overlap of two bosonic CSs can be shown \cite{grossmann} to be
\begin{equation}
\braket{z_a}{z_b}=N(z_a)N(z_b)\left(1+\sum_i^{M-1}z_{a,i}^*z_{b,i}\right)^S
\end{equation}
This also determines the normalisation function as
$N(z)=\left(1+\sum z_i^*z_i\right)^{-\frac{S}{2}}$


\subsubsection{Matrix element of bosonic operator sequences}
Consider a normal-ordered product sequence of $x$ bosonic creation and $x$ bosonic annihilation operators. Its matrix element for two arbitrary CSs can be shown \cite{grossmann} to be
\begin{equation}
\mel{z_a}{\hat{b}\hc_{\rho_1}\dots\hat{b}\hc_{\rho_x}\hat{b}_{\sigma_1}\dots\hat{b}_{\sigma_x}}{z_b}=z_{a,\rho_1}^*\dots z_{a,\rho_x}^*z_{b,\sigma_1}\dots z_{b,\sigma_x} \braket{z_a^{(x)}}{z_b^{(x)}}
\end{equation}
where the \textit{reduced overlap} is defined as
\begin{equation}
\braket{z_a^{(x)}}{z_b^{(x)}}=N(z_a)N(z_b)\left(1+\sum_i^{M-1}z_{a,i}^*z_{b,i}\right)^{S-x}
\end{equation}
i.e. the reduced unnormalised CSs behave as if their associated total particle number was reduced by $x$.


\subsection{Results}

	
	
	
	\section{Fermionic $SU(M)$ coherent states}

\subsection{Construction of the unnormalised state}

\subsection{Overlap and normalisation}

\subsection{Fermionic operator sequence matrix element}

\subsection{Connection to molecular electronic structure}
quote the two body and one body integrals matey

\subsection{How to get that sweet sweet ground state}
	
	
	
	\chapter{Outlook}
	
	we want that nobel ngl
	
	
	
	%\clearpage
	\begin{thebibliography}{10}

	\addcontentsline{toc}{chapter}{Bibliography}
	
	\bibitem{curse_of_dimensionality}
	Shalashilin, D. V. (2011), Multiconfigurational Ehrenfest approach to quantum coherent dynamics in large molecular systems. \textit{Faraday Discussions}, \textbf{153}, pp. 105--116
	
	\bibitem{harmonic_classical_states}
	Schrodinger, E. (1926), Der stetige übergang von der Mikro-zur Makromechanik. \textit{Naturwiss}. 14, pp. 664-666. Translated into English in Schrodinger, E. (1928), \textit{Collected Papers in Wave Mechanics}. 1st edn. London: Blackie \& Son, pp. 41--44
	
	\bibitem{field_coherent_states}
	Glauber, R. J. (1963), Coherent and Incoherent States of the Radiation Field. \textit{Phys. Rev.}, \textbf{131}, 6, pp. 2766--2788
	
	\bibitem{no_unity}
	Klauder, J. R., Skagerstam, B. S. (1985), \textit{Coherent States: Applications in Physics and Mathematical Physics}. 1st eng. edn. Singapore: World Scientific.
	
	\bibitem{perelomov_og}
	Perelomov, A. M. (1972), Coherent states for arbitrary Lie group. \textit{Commun. Math. Phys.}, \textbf{26}, pp. 222--236
	
	\bibitem{gilmore_og}
	Gilmore, R. (1972), Geometry of symmetrized states. \textit{Ann. Phys.}, \textbf{74}, 2, pp. 391--463
	
	\bibitem{ZFG}
	Zhang, W. M., Feng, D. H., Gilmore, R. (1990), Coherent states: Theory and some applications. \textit{Rev. Mod. Phys.}, \textbf{62}, pp. 867--927
	
	\bibitem{aguiar}
	Viscondi, T. F., Grigolo, A., de Aguiar, M. A. M. (2015), Semiclassical Propagator in the Generalized Coherent-State Representation. \href{https://doi.org/10.48550/arXiv.1510.05952}{arXiv:1510.05952 \textbf{[quant-ph]}}
	
	\bibitem{optical_lattices}
	Block, I., Dalibard, J., Zwerger, W. (2008), Many-body physics with ultracold gases. \textit{Rev. Mod. Phys.}, \textbf{80}, pp. 885--964
	
	\bibitem{grossmann}
	Qiao, Y., Grossmann, F. (2021), Exact variational dynamics of the multimode Bose-Hubbard model based on $SU(M)$ coherent states. \textit{Phys. Rev. A}, \textbf{103}, 042209
	
	\bibitem{green}
	Green, J. A., Shalashilin, D. V. (2019), Simulation of the quantum dynamics of indistinguishable bosons with the method of coupled coherent states. \textit{Phys. Rev. A}, \textbf{100}, 013607
	
	\bibitem{gellmann}
	Bertlmann, R. A., Krammer, P. (2008), Bloch vectors for qudits. \href{ 	
https://doi.org/10.48550/arXiv.0806.1174}{arXiv:0806.1174 \textbf{[quant-ph]}}

	\bibitem{buonsante}
	Buonsante, P., Penna, V.(2008), Some remarks on the coherent-state variational approach to nonlinear boson models. \textit{J. Phys. A: Math. Theor.}, \textbf{41}, 175301
	
	\bibitem{sampling_algorithm}
	Grigolo, A., Viscondi, T. F., de Aguiar, M. A. M. (2016), Multiconfigurational quantum propagation with trajectory-guided generalized coherent states. \textit{J. Chem. Phys.}, \textbf{144}, 094106

	\end{thebibliography}
	\bibliographystyle{unsrt}
	
	\appendix
	
	\chapter{Auxilary theorems}
This appendix presents mathematical results which were used to derive the mathematical properties of bosonic and fermionic coherent states, but which were chosen to be omitted from the main body of text due to their abstract mathematical nature.

The author does not claim originality of these theorems or their proofs, but chooses to include them, as he was unable to find them in existing literature. The work presented in this appendix is the author's, except for mathematical identities which are stated explicitly.

\section{counting minors}
	
	
	
	
	

	
	
\end{document}