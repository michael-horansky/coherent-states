\documentclass[12pt]{report}
\usepackage[a4paper, total={6.5in, 8in}]{geometry}
\usepackage{xargs}
\usepackage{amsmath,amssymb}
\usepackage{hyperref}
\usepackage{physics}
\usepackage{graphicx}

\newcommand{\sgn}{\text{sgn}}
\newcommand{\seq}[1]{\langle #1\rangle}
\newcommand{\asc}[1]{\upharpoonleft #1 \upharpoonright}
\newcommand{\hc}{^\dagger}
\newcommand{\inv}{^{-1}}
\newcommand{\Sym}{\text{Sym}}
\newcommand{\normord}[1]{:\mathrel{\mspace{2mu}#1\mspace{2mu}}:}

\newtheorem{theorem}{Theorem}[section]
\newtheorem{lemma}[theorem]{Lemma}

\begin{document}

	\title{Bosonic and fermionic particle-preserving coherent states, their dynamics and applications\\\hfill\\Transfer Report}
	\author{Michal Horanský}
	\maketitle
	
	\chapter*{Abstract}
	Coherent states-based methods have been previously developed and utilised for particle-preserving bosonic systems, such as the Bose-Hubbard model. This work is divided into two parts: Firstly, a new method is proposed for the bosonic systems, where the basis coherent states are propagated separately to create a frozen dynamical basis, which lowers the time-complexity significantly. Second, an analogous method is proposed for particle-preserving fermionic systems, and particularly the electronic structure of a molecule, for which the goal of this method is to find the ground state. For this, multiple methods are outlined, and relevant properties of the fermionic coherent states are obtained.
	
	
	\tableofcontents
	
	
	\chapter{Background}
	
	To find the dynamics of a given system, one typically needs to solve the Schrodinger equation given by the system's Hamiltonian. For complex systems which can be found in nature, an analytic solution rarely exists. To solve this partial differential equation computationally, one needs to chose a basis for the Hilbert space and, since it typically is infinite or even continuous, sample it to obtain a basis sample onto which the initial state is decomposed and on which it is propagated. This sampling choice is always limiting and must be carefully justified, as omitting elements of the Hilbert space may render the analysis inaccurate.
	
	Coherent states are particular states in the Hilbert space which follow classical trajectories, remain coherent, and minimise position-momentum uncertainty. Choosing them as the basis, a very small sample is typically suitable to capture the physics of a wide variety of systems and starting conditions \cite{grossmann}. This is an example of a semi-classical approximation.
	
	\section{Utility and scope of semi-classical methods}
get that adiabatic shit outta here lil bro
	
\subsection{Bosonic system 1: Bose-Hubbard model}
he bossin
	
\subsection{Bosonic system 2: Displaced harmonic trap}
me when i quantise the modes
	
\subsection{Fermionic system: molecular electronic structure}
ground state would be nice aha

	
	
	
	\section{Coherent states: a hundred year-long history}
Here we write that sweet history of coherent states
	
\subsection{Schrodinger: the harmonic oscillator}
Truly coherent, mm
	
\subsection{Glauber: field coherent states}
we exponentiating in this mofo
	
\subsection{Zhang, Feng, Gilmore: coupled coherent states}
me when i group

	
	
	
	\section{Mathematical approach to CS-based methods}\label{sec:mathematical_approach}
In this section, we firstly discuss the mathematical properties of generalised field coherent states and their ensembles, and then the particular construction of $SU(M)$ coherent states.
	
\subsection{Topology of the CS parameter space}
This subsection is a summary of relevant information as included by Viscondi, Grigolo, and de Aguiar in their review of generalised field coherent states \cite{aguiar}.

When applying the quotient-space displacement operator on the reference state, the resulting states are no longer normalised. We will adopt the notation of Viscondi, Grigolo, and de Aguiar, and denote unnormalised elements of the Hilbert space as bras and kets with curly brackets. Then, the unnormalised coherent state is directly obtained as
\begin{equation}
\unnormket{z}=\hat{D}(z)\ket{\phi_0}
\end{equation}
and a normalised coherent state is constructed by scaling by a normalisation factor:
\begin{equation}
\ket{z}=N(z^*, z)\unnormket{z}\qq{where}N(z_a^*, z_b)=\unnormbraket{z_a}{z_b}^{-\frac{1}{2}}
\end{equation}
The quotient space can be characterised by the following metric:
\begin{equation}
g(z_a^*, z_b)=\pdv{\ln\unnormbraket{z_a^*}{z_b}}{z_b}{z_a^*}
\end{equation} 
Then, the coherent states remain coherent under time evolution, with the parameter $z$ evolving according to a Hamiltonian equation on a curved manifold:
\begin{equation}
\dot{z}_i=-i\xi^T_{ij}(z^*, z)\pdv{H(z^*, z)}{z_j^*}
\end{equation}
where $H(z^*, z)=\mel{z}{\hat{H}}{z}$ is the effective Hamiltonian and $\xi(z^*, z)$ is the inverse of the quotient-space metric $g(z^*, z)$.

	
\subsection{Fully variational equations of motion}
Here we employ the paradigm discussed in previous sections which encompasses the heart of this kind of semi-classical approaximation: a limited sample of classical trajectories. We will approximate our Hilbert space by a discrete sample of $N$ coherent states, each characterised by its parameters $z_a$, which evolve in time. Onto this basis sample, we decompose an arbitrary wavestate $\ket{\Psi(t=0)}$ whose dynamics we are interested in:
\begin{equation}
\ket{\Psi(t=0)}=\sum_{a=1}^N A_a(t=0)\unnormket{z_a(t=0)}
\end{equation}
Note that I opt to use the unnormalised coherent states as the basis, rather than their normalised counterparts. This does not affect the physicality of the solution, which shall remain normalised, and it serves to simplify the equations of motion by omitting the normalisation factors $N(z_a)$.

Now, we can apply the Schrodinger Lagrangian $\hat{L}=i\dv{t}-\hat{H}$ onto our system of free coordinates $(A_a, z_a)$ to obtain a system of first-order equations of motion, which can be computationally solved to obtain the trajectories of the basis, the decomposition coefficient evolution, and hence the evolution of the initial wavestate. This approach is a slight modification of the approach used by Qiao and Grossmann in \cite{grossmann}.

Let us begin by obtaining the derivatives of unnormalised coherent states:
\begin{align}
\pdv{z_j}\unnormket{z}&=\pdv{z_j}\exp(z_i\hat{T}_i)\ket{\phi_0}=\hat{T}_j\unnormket{z}\\
\dv{t}\unnormket{z}&=\dv{t}\exp(z_i\hat{T}_i)\ket{\phi_0}=\dot{z}_i\hat{T}_i\unnormket{z}
\end{align}


\subsection{$SU(M)$ coherent states}
	
	
	
	
	
	\chapter{Current work}
	In this chapter I outline the application of the mathematical construction of $SU(M)$ coherent states in particular to bosonic systems (such as the ones described in Sec. \ref{sec:bosonic1}, \ref{sec:bosonic2}) and fermionic systems (such as the one described in Sec. \ref{sec:fermionic1}).
	
	\section{Bosonic $SU(M)$ coherent states}
	
\subsection{Construction}
transitioning into a better person rn
	
\subsection{Results}

	
	
	
	\section{Fermionic $SU(M)$ coherent states}

\subsection{Construction of the unnormalised state}
In this subsection, I describe the properties of $SU(M)$ coherent states constructed on fermionic annihilation and creation operators. This construction is well-known in literature; while not equivalent, it is analogous to the construction described by Grigolo, Viscondi, and de Aguiar in \cite[Sec. 4]{sampling_algorithm}.

\subsubsection{Reference state and displacement operator}

Anticipating the application of this ansatz to finding the ground state of a complex system, the reference state is chosen as the Slater determinant of the $S$ lowest-energy modes; in second quantisation, this is succintly written as
\begin{equation}
\ket{\phi_0}=\ket{1,\dots 1, 0, \dots 0}\qq{where the first $S$ modes are occupied}
\end{equation}
The only transition operators $\hat{T}_{ij}=\hat{f}\hc_i\hat{f}_j$ which do not leave the reference state invariant up to a scalar factors are the ones with $i\in\{S+1\dots M\}, j\in\{1\dots S\}$. For convenience we define two sets of indices
\begin{equation}
\pi_1=\{1, 2\dots S\}\qquad \pi_0=\{S+1, S+2 \dots M\}
\end{equation}
Then the displacement operator becomes
\begin{equation}
\hat{D}(Z)=\exp(\sum_{i\in\pi_0}\sum_{j\in\pi_1}Z_{ij}\hat{f}\hc_i\hat{f}_j)
\end{equation}
and the fermionic $SU(M)$ coherent state is
\begin{equation}
\ket{Z}=N(Z)\exp(\sum_{i\in\pi_0}\sum_{j\in\pi_1}Z_{ij}\hat{f}\hc_i\hat{f}_j)\ket{\phi_0}
\end{equation}
where $Z$ is a complex $(M-S \cross S)$ matrix.

\subsubsection{Decomposition into the occupancy basis}
We use Eq. \ref{eq:hadamard sum}. Firstly, the reference state operator is
\begin{equation}
\hat{\phi}_0=\hat{f}\hc_{\seq{\pi_1}}
\end{equation}
where $\seq{\pi_1}$ denotes the ascending sequence formed from elements of the set $\pi_1$. We also have the operator
\begin{equation}
\hat{F}=\sum_{i\in\pi_0}\sum_{j\in\pi_1}Z_{ij}\hat{f}\hc_i\hat{f}_j
\end{equation}
Then the repeated commutator may be found to be
\begin{equation}
\comm{\hat{D}(Z)}{\hat{\phi}_0}_x=
	(-1)^{\frac{1}{2}x(x+1)}x!\sum_{\seq{a}\in\Gamma_x\seq{\pi_0}}\sum_{\seq{b}\in\Gamma_x\seq{\pi_1}}\det(Z_{\seq{a},\seq{b}})\hat{f}\hc_{\seq{\pi_1-\{b\}+\{a\}}}
\end{equation}
where $\Gamma_x\seq{S}$ denotes all increasing subsequences of length $x$ of index sequence $\seq{S}$. In other words, the unnormalised coherent state
\begin{equation}
\unnormket{Z}=\sum_{r=0}^{\min(S,M-S)}(-1)^{\frac{1}{2}r(r+1)}\sum_{\seq{a}\in\Gamma_r\seq{\pi_0}}\sum_{\seq{b}\in\Gamma_r\seq{\pi_1}}\det(Z_{\seq{a},\seq{b}})\ket{\pi_1-\{b\}+\{a\}}
\end{equation}
is a superposition of occupancy basis states such that the component with particles excited from initially occupied indices $\seq{b}$ to initially unoccupied indices $\seq{a}$ is weighted by the minor of the parameter matrix obtained by selecting the rows labelled by $\seq{a}$ and columns labelled by $\seq{b}$, up to a sign factor.

\subsection{Overlap and normalisation}
The overlap of two unnormalised fermionic coherent states may be expressed as
\begin{equation}
\unnormbraket{Z_a}{Z_b}=\sum_{r=0}^{\min(S,M-S)}\sum_{\seq{a}\in\Gamma_r\seq{\pi_0}}\sum_{\seq{b}\in\Gamma_r\seq{\pi_1}}\det((Z_a\hc)_{\seq{b},\seq{a}})\det((Z_b)_{\seq{a},\seq{b}})
\end{equation}
where we may use the identity
\begin{equation}
\det((Z_a\hc)_{\seq{b},\seq{a}})\det((Z_b)_{\seq{a},\seq{b}})=\det((Z_a\hc Z_b)_{\seq{b},\seq{b}})=\det((Z_bZ_a\hc)_{\seq{a},\seq{a}})
\end{equation}
to express this as a sum of all principal minors of a square matrix with sign alternating between their ranks. Using Lemma \ref{lemma:sum of constrained principal minors} yields the overlap
\begin{equation}
\unnormbraket{Z_a}{Z_b}=\det(I+Z_a\hc Z_b)=\det(I+Z_b Z_a\hc)
\end{equation}
which also determines the normalisation factor
\begin{equation}
N(Z)=\frac{1}{\sqrt{\det(I+Z\hc Z)}}
\end{equation}
which agrees with the result in \cite[Eq. 2.31]{sampling_algorithm}.

\subsection{Fermionic operator sequence matrix element}
We now wish to evaluate the matrix element
\begin{equation}
S=\unnormmel{Z_a}{\hat{f}\hc_{\rho_1}\dots\hat{f}\hc_{\rho_x}\hat{f}_{\rho'_1}\dots\hat{f}_{\rho'_x}}{Z_b}
\end{equation}
I was not able to find a general result of this form in existing literature, and thus I now present what is, according to the best of my knowledge, my original work and main theoretical contribution to the mathematical construction of $SU(M)$ fermionic coherent states.

First, we express the sequences $\rho, \rho'$ as permutations of strictly descending and ascending sequences, respectively: $\rho=P_-\seq{\rho}^-,\rho'=P_+\seq{\rho'}^+$. Replacing the original sequences with the ordered versions introduces a factor of $\sgn(P_-)\sgn(P_+)$. We now partition the sequences into segments constructed of elements of $\pi_1$ and $\pi_0$:
\begin{equation}
S=\sgn(P_-)\sgn(P_+)\unnormmel{Z_a}{\hat{f}\hc_{\seq{\tau}^-}\hat{f}\hc_{\seq{\sigma}^-}\hat{f}_{\seq{\sigma'}}\hat{f}_{\seq{\tau'}}}{Z_b}
\end{equation}
where
\begin{equation}
\rho = \sigma \cup \tau\qq{where}\sigma\in\pi_1,\tau\in\pi_0\qq{and}\rho' = \sigma' \cup \tau'\qq{where}\sigma'\in\pi_1,\tau'\in\pi_0
\end{equation}
Let us use $\eta_x(\chi)$ to denote the number of elements in set $\chi$ smaller than $x$. This is particularly useful to keep track of the Jordan-Wigner string. Taking the matrix element as an overlap of $\hat{f}\hc_{\seq{\sigma}^+}\hat{f}\hc_{\seq{\tau}^+}\unnormket{Z_a}$ and $\hat{f}_{\seq{\sigma'}}\hat{f}_{\seq{\tau'}}\unnormket{Z_b}$ and using the decomposition into the occupancy basis, we obtain
\begin{multline}
	S = \sgn(P_-)\sgn(P_+)\sum_{r=\abs{\tau}}^{\min(S-\abs{\sigma}, M-S)}(-1)^{\frac{1}{2}r(r+1)}\sum_{r'=\abs{\tau'}}^{\min(S-\abs{\sigma'}, M-S)}(-1)^{\frac{1}{2}r'(r'+1)}\\
	\sum_{\seq{a}\in\Gamma_{r-\abs{\tau}}\seq{\pi_0-\tau}}\sum_{\seq{b}\in\Gamma_{r}\seq{\pi_1-\sigma}}\sum_{\seq{a'}\in\Gamma_{r'-\abs{\tau'}}\seq{\pi_0-\tau'}}\sum_{\seq{b'}\in\Gamma_{r'}\seq{\pi_1-\sigma'}}(-1)^{\abs{\tau}(S-r)+\frac{1}{2}\abs{\tau}(\abs{\tau}-1)+\sum_i\eta_{\tau_i}(\seq{a})}\\
	(-1)^{-\abs{\sigma}+\sum_i(\sigma_i+\eta_{\sigma_i}(\seq{b}))}(-1)^{\abs{\tau'}(S-r')+\frac{1}{2}\abs{\tau'}(\abs{\tau'}-1)+\sum_i\eta_{\tau'_i}(\seq{a'})}(-1)^{-\abs{\sigma'}+\sum_i(\sigma'_i+\eta_{\sigma'_i}(\seq{b'}))}\\
	\det((Z_a\hc)_{\seq{b},\seq{a\cup\tau}})\det((Z_b)_{\seq{a'\cup\tau'},\seq{b'}})\braket{\pi_1\cup a-b\cup\sigma}{\pi_1\cup a'-b'\cup\sigma'}
	\end{multline}
	The occupancy basis overlap is equivalent to
	\begin{equation}
	\braket{\pi_1\cup a-b\cup\sigma}{\pi_1\cup a'-b'\cup\sigma'}=\delta_{\seq{a},\seq{a'}}\delta_{\seq{b\cup\sigma},\seq{b'\cup\sigma'}}\delta_{r-\abs{\tau},r'-\abs{\tau}}\delta_{r+\abs{\sigma},r'+\abs{\sigma}}
	\end{equation}
	Note that, since $\abs{\sigma}+\abs{\tau}=\abs{\sigma'}+\abs{\tau'}$ unless the overlap vanishes due to mismatched total number of particles, the final two Kronecker deltas for $r,r'$ are equivalent.
	
	We now take
	\begin{align}
	\gamma &= r - \abs{\tau} = r' - \abs{\tau'}\qq{so that}r=\gamma+\abs{\tau},r'=\gamma+\abs{\tau'}\\
	\seq{\alpha}&\in\Gamma_{\gamma}\seq{\pi_0-\tau\cup\tau'}\qq{so that}\seq{a}=\seq{a'}=\seq{\alpha}\\
	\seq{\beta}&\in\Gamma_{\gamma+\abs{\tau}-\abs{\sigma'-\sigma\cap\sigma'}}\seq{\pi_1-\sigma\cup\sigma'}\qq{so that}\seq{b}=\seq{\beta\cup\sigma' - \sigma\cap\sigma'},\seq{b'}=\seq{\beta\cup\sigma - \sigma\cap\sigma'}
	\end{align}
	where $\abs{\tau}-\abs{\sigma'}=\abs{\tau'}-\abs{\sigma}$ and the construction of $\seq{b},\seq{b'}$ omits $\sigma\cap\sigma'$, since the terms with $\seq{b}$ containing any element in $\sigma$ vanish (same for $\seq{b'}$ and $\sigma'$).
	
	Substituing $r, r', \seq{a},\seq{a'}, \seq{b},\seq{b'}$ and using simple algebraic manipulation we can show that, for terms with non-vanishing Kronecker deltas, the total sign simplifies significantly. Denoting $\varsigma=\sigma-\sigma\cap\sigma',\varsigma'=\sigma'-\sigma\cap\sigma'$, the overlap can be written as
	\begin{multline}
	=(-1)^{S(\abs{\tau}+\abs{\tau'})+(\abs{\varsigma}-1)(\abs{\varsigma'}-1)+1+\sum\seq{\varsigma}+\sum\seq{\varsigma'}+\sum_i\eta_{(\sigma\cap\sigma')_i}(\seq{\varsigma\cup\varsigma'})}\sum_{\gamma=0}\sum_{\seq{\alpha}\in\Gamma_{\gamma}\seq{\pi'_0-\tau\cup\tau'}}\sum_{\seq{\beta}\in\Gamma_{\gamma+\abs{\tau}-\abs{\varsigma'}}\seq{\pi'_1-\sigma\cup\sigma'}}\\
	(-1)^{\sum_i\eta_{\varsigma_i}(\seq{\beta})+\sum_i\eta_{\varsigma'_i}(\seq{\beta})+\sum_i\eta_{\tau_i}(\seq{\alpha})+\sum_i\eta_{\tau'_i}(\seq{\alpha})}\det((Z_a\hc)^{(\text{r.} \sigma, \text{c.} \tau'-\tau\cap\tau')}_{\seq{\beta\cup\varsigma'}, \seq{\alpha\cup\tau}})\det((Z_b)^{(\text{r.} \tau-\tau\cap\tau', \text{c.} \sigma')}_{\seq{\alpha\cup\tau'},\seq{\beta\cup\varsigma}})
	\end{multline}
	where the superscript $(\text{r.} X), (\text{c.} X)$ means omitting the rows or columns specified by the set of indices $X$, and where the summation over $\gamma,\seq{\alpha},\seq{\beta}$ is such that all square submatrices of $(Z_a\hc)^{(\text{r.} \sigma)},(Z_b)^{(\text{c.} \sigma')}$ are present in the sum, as denoted by the apostrophed $\pi_1,\pi_0$, which represents the omission of indices corresponding to the removed rows and columns.
	
	We now choose to permute the rows and columns of $(Z_a\hc)^{(\text{r.} \sigma)},(Z_b)^{(\text{c.} \sigma')}$ as to bring the rows and columns which are included in every submatrix in every term of the sum to the lowest-index position. This introduces an extra sign factor to the determinant, which cancels the second sign term in the sum above. Formally
	\begin{align}
	X &= \mqty(
		(Z_a\hc)_{\seq{\varsigma'},\seq{\tau}} & (Z_a\hc)^{(\text{c.} \tau\cup\tau')}_{\text{r.} \seq{\varsigma'}}\\
		(Z_a\hc)^{(\text{r.} \sigma\cup\sigma')}_{\text{c.} \seq{\tau}} & (Z_a\hc)^{(\text{r.} \sigma\cup\sigma', \text{c.} \tau\cup\tau')}
	)\\
	Y &= \mqty(
		(Z_b)_{\seq{\tau'},\seq{\varsigma}} & (Z_b)^{(\text{c.} \sigma\cup\sigma')}_{\text{r.} \seq{\tau'}}\\
		(Z_b)^{(\text{r.} \tau\cup\tau')}_{\text{c.} \seq{\varsigma}} & (Z_b)^{(\text{r.} \tau\cup\tau', \text{c.} \sigma\cup\sigma')}
	)
	\end{align}
	
	The resulting expression is exactly in the form which is treated by Lemma \ref{lemma:asymmetrically constrained sum of complementary minors}. Hence, if $\abs{\tau}\leq\abs{\tau'}$, we have
	\begin{align} \label{eq: general reduced overlap determinant}
	&\mel{Z_a}{\hat{f}\hc_{\seq{\tau}^-}\hat{f}\hc_{\seq{\sigma}^-}\hat{f}_{\seq{\sigma'}}\hat{f}_{\seq{\tau'}}}{Z_b}=N(Z_a)N(Z_b)(-1)^{S(\abs{\tau}+\abs{\tau'})+(\abs{\varsigma}-1)(\abs{\varsigma'}-1)+1+\sum\seq{\varsigma}+\sum\seq{\varsigma'}+\sum_i\eta_{(\sigma\cap\sigma')_i}(\seq{\varsigma\cup\varsigma'})}\nonumber\\
	&(-1)^{\abs{\tau'}(1+\abs{\tau'}-\abs{\tau})}\det(\mqty{
		0_{\abs{\tau'},\abs{\tau}} & (Z_b)_{\seq{\tau'},\seq{\varsigma}} & (Z_b)^{(\text{c.} \sigma\cup\sigma')}_{\text{r.} \seq{\tau'}}\\
		(Z_a\hc)_{\seq{\varsigma'},\seq{\tau}} & (Z_a\hc)^{(\text{c.} \tau\cup\tau')}_{\text{r.} \seq{\varsigma'}}(Z_b)^{(\text{r.} \tau\cup\tau')}_{\text{c.} \seq{\varsigma}} & (Z_a\hc)^{(\text{c.} \tau\cup\tau')}_{\text{r.} \seq{\varsigma'}}(Z_b)^{(\text{r.} \tau\cup\tau', \text{c.} \sigma\cup\sigma')}\\
		(Z_a\hc)^{(\text{r.} \sigma\cup\sigma')}_{\text{c.} \seq{\tau}} & (Z_a\hc)^{(\text{r.} \sigma\cup\sigma', \text{c.} \tau\cup\tau')}(Z_b)^{(\text{r.} \tau\cup\tau')}_{\text{c.} \seq{\varsigma}} & I + (Z_a\hc)^{(\text{r.} \sigma\cup\sigma', \text{c.} \tau\cup\tau')}(Z_b)^{(\text{r.} \tau\cup\tau', \text{c.} \sigma\cup\sigma')}
	})
	\end{align}
	If $\abs{\tau}\geq\abs{\tau'}$, we have $\abs{\varsigma}\leq\abs{\varsigma'}$, which allows us to apply Lemma \ref{lemma:asymmetrically constrained sum of complementary minors} to obtain the expression
	\begin{align} \label{eq: general reduced overlap determinant alt}
	&\mel{Z_a}{\hat{f}\hc_{\seq{\tau}^-}\hat{f}\hc_{\seq{\sigma}^-}\hat{f}_{\seq{\sigma'}}\hat{f}_{\seq{\tau'}}}{Z_b}=N(Z_a)N(Z_b)(-1)^{S(\abs{\tau}+\abs{\tau'})+(\abs{\varsigma}-1)(\abs{\varsigma'}-1)+1+\sum\seq{\varsigma}+\sum\seq{\varsigma'}+\sum_i\eta_{(\sigma\cap\sigma')_i}(\seq{\varsigma\cup\varsigma'})}\nonumber\\
	&(-1)^{\abs{\varsigma'}(1+\abs{\varsigma'}-\abs{\varsigma})}\det(\mqty{
		0_{\abs{\varsigma'},\abs{\varsigma}} & (Z_a\hc)_{\seq{\varsigma'},\seq{\tau}} & (Z_a\hc)^{(\text{c.} \tau\cup\tau')}_{\text{r.} \seq{\varsigma'}}\\		
		(Z_b)_{\seq{\tau'},\seq{\varsigma}} & (Z_b)^{(\text{c.} \sigma\cup\sigma')}_{\text{r.} \seq{\tau'}}(Z_a\hc)^{(\text{r.} \sigma\cup\sigma')}_{\text{c.} \seq{\tau}} & (Z_b)^{(\text{c.} \sigma\cup\sigma')}_{\text{r.} \seq{\tau'}}(Z_a\hc)^{(\text{r.} \sigma\cup\sigma', \text{c.} \tau\cup\tau')}\\	
		(Z_b)^{(\text{r.} \tau\cup\tau')}_{\text{c.} \seq{\varsigma}} & (Z_b)^{(\text{r.} \tau\cup\tau', \text{c.} \sigma\cup\sigma')}(Z_a\hc)^{(\text{r.} \sigma\cup\sigma')}_{\text{c.} \seq{\tau}} & I + (Z_b)^{(\text{r.} \tau\cup\tau', \text{c.} \sigma\cup\sigma')}(Z_a\hc)^{(\text{r.} \sigma\cup\sigma', \text{c.} \tau\cup\tau')}
	})
	\end{align}
	Note that for the case $\abs{\tau}>\abs{\tau'}$, we can simply take
	\begin{equation}
	\mel{Z_a}{\hat{f}\hc_{\seq{\tau}^-}\hat{f}\hc_{\seq{\sigma}^-}\hat{f}_{\seq{\sigma'}}\hat{f}_{\seq{\tau'}}}{Z_b}=\mel{Z_b}{\hat{f}\hc_{\seq{\tau'}^-}\hat{f}\hc_{\seq{\sigma'}^-}\hat{f}_{\seq{\sigma}}\hat{f}_{\seq{\tau}}}{Z_a}^*
	\end{equation}
	so that we can always use Eq. \ref{eq: general reduced overlap determinant} as the standard expression.

\subsection{Connection to molecular electronic structure}
An example of a fermionic particle-preserving system, as stated in Sec. \ref{sec:fermionic1}, is the molecular electronic structure when no chemical process is occuring. Formally, we may index all the atomic orbitals of the constituent atoms such that the occupied orbitals are included in $\pi_1$, unoccupied orbitals are included in $\pi_0$, and within both of these sets, the orbitals are ordered by their self-energy (it is important to take notice of spin degeneracy, as each orbitals actually corresponds to two modes). This is a gross-structure model, governed by a second-order Hamiltonian
\begin{equation}
\hat{H}=V^{(1)}_{\alpha\beta}\hat{f}\hc_\alpha\hat{f}_\beta+\frac{1}{2}V^{(2)}_{\alpha\beta\gamma\delta}\hat{f}\hc_\alpha\hat{f}\hc_\beta\hat{f}_\gamma\hat{f}_\delta
\end{equation}
where $V^{(1)}$ is the single-body energy tensor, here comprising of the kinetic term and electron-nucleus interaction, and $V^{(2)}$ is the two-body energy tensor, comprising of the electron-electron exchange integrals (care must be exercised, as in Mulliken notation, the quadratic term is not normal-ordered, and thus an extra diagonal term is introduced). These tensors are typically obtained by numerical calculation; I use the framework of Python, and specifically the PySCF package to calculate them for an arbitrary molecule with an arbitrary geometry and arbitrary basis for the atomic orbitals.

\subsection{How to calculate the ground state}
Once the second-quantised framework of $SU(M)$ coherent states is applied to our desired molecule, we have multiple options on how to quickly approximate the ground state. Below are listed a few methods we are in the process of implementing for the purpose of benchmarking. All of these methods use the state $\ket{Z=0}$, i.e. the "atomic orbitals populated as if outside of a molecule" state, as a starting point. This state shall be, for convenience, referred to as the "null state".

The first two methods specifically build on the idea of restricting the full Hilbert space into a much smaller subspace by quick energy considerations, on which full diagonalisation is then performed. The null state is taken as belonging to this small subspace, and the ground state is assumed to be "close" to the null state (formally, their overlap is assumed to be close to unity). In the first two of the three methods presented, the low-energy subspace is formed on a basis consisting of fermionic $SU(M)$ coherent states, and the ground state estimate is expressed as their superposition (which is not necessarily a pure coherent state in itself).

\subsubsection{Random walk around the null state}
The simplest algorithm for sampling the low-energy subspace is a random walk around the null state, onto which conditioning may be imposed. Formally, in each step a candidate state is considered by randomly sampling the coherent-state parameter space in the vicinity of the null state, and the candidate is either included in the full sample or rejected based on an arbitrary criterion (for example, whether its expected energy is under a certain treshold compared to the null state energy, or whether the eigenvalues of the Hamiltonian matrix on the full sample change in a favourable way). Then, this step is repeated either until the sample reaches a certain size, the estimated ground state energy no longer decreases dramatically, or a certain number of rejections occured. Taking the sample $\ket{Z_a}, a=1\dots N$ as an un-orthogonal basis of the low-energy subspace, constructing the Hamiltonian matrix $H_{ij}=\mel{Z_i}{\hat{H}}{Z_j}$, diagonalising it, choosing the eigenvector with the lowest associated eigenvalue and expressing it in the occupancy basis then yields the estimate of the ground state.

\subsubsection{Sampling the trajectory of the null state}
Another method for selecting a basis of the low-energy subspace exploits the assumption that the null state has low energy. Then its trajectory under proper time evolution traces states with the same low energy. Sampling this trajectory at regular time intervals then yields the basis of a low-energy subspace, diagonalising the Hamiltonian on which then yields the estimate of the ground state.

\subsubsection{Imaginary time propagation}
This is a well-known method \cite{imaginary_timeprop} based on the fact that a superposition of energy eigenstates propagated in time has the phase of each component change at a rate proportional to the associated energy, with the amplitude remaining constant. Propagating along the imaginary time $-it$ then instead has the amplitude of each component decay exponentially at a rate given by the component's energy. Eventually, after a sufficiently prolonged propagation, only the lowest-energy state is present in the decomposition. Assuming the null state contains the ground state as a component, propagating the null state along $-it$ should then converge to the ground state. This method is theoretically guaranteed to converge to the true ground state, but the rate of convergence may be slow if the low-excitation energy eigenstates have comparable energy levels to the ground state.
	
	
	
	\chapter{Outlook}
	
	In this chapter the scope of work which has been done is summarised and an outline of the next steps of the research is presented.
	
	\section{Summary of original work done}
	
	During the first year of this doctorate degree, I have done the following original work:
	\begin{itemize}
	\item I have generalised the matrix form of the fully-variational equations of motion on a discrete CS sample as quoted by Qiao and Grossmann \cite[App. B]{grossmann}, which is also applicable to fermionic coherent states.
	\item Based on the suggestion by my supervisor Prof Dmitry Shalashilin, I have built a framework in Python 3 for bosonic $SU(M)$ dynamics on an arbitrary second-order Hamiltonian for both the fully variational method and the new separate-basis-trajectory method, and implemented the Bose-Hubbard and the displaced harmonic trap systems in this framework.
	\item Using this framework, I have achieved convergence for two- and three-mode systems for the Bose-Hubbard model on a scale which matches that of the work of Qiao and Grossmann, demonstrating the utility of the separate-basis-trajectory method. I have also extended the scope of this investigation onto problems with initial wavefunctions which are not pure bosonic coherent states.
	\item I have derived the general matrix element of a product sequence of fermionic creation and annihilation operators for fermionic $SU(M)$ coherent states, which allows the evaluation of matrix elements of arbitrary particle-preserving operators, including the Hamiltonian.
	\item I have built the skeleton of a framework in Python 3 using PySCF to translate the single-electron and electron-exchange energy integrals for an arbitrary molecule into a mode system onto which fermionic $SU(M)$ coherent state-based methods may be applied.
	\end{itemize}
	
	\section{Research plan for the duration of the doctorate programme}
	I propose the following tasks for the remaining duration of my doctorate programme to be done within this research project:
	\begin{itemize}
	\item Finishing the bosonic dynamics for separate basis trajectories (estimated duration: 2 months). This entails
		\begin{itemize}
		\item Implementing arbitrary initial wavefunction sampling techniques and testing various conditioning methods to maximise numerical stability
		\item Scaling the mode and particle numbers to see the limits of this method
		\item For high mode and particle numbers, it is no longer possible to calculate the "true" solution on the full occupancy basis; comparison data must be collected by other techniques, e.g. MCSCF.
		\end{itemize}
	\item Obtaining a succesful estimate for the ground state of a complex molecule using $SU(M)$ fermionic coherent states (estimated duration: 2 months).
	\item Further investigation of the mathematical properties of $SU(M)$ fermionic coherent states, with the focus on lowering the time complexity of calculating the Hamiltonian matrix elements (estimated duration: 3 months).
	\item Benchmarking different methods for obtaining the molecular ground state (estimated duration: 3 months).
	\item Investigating alternative constructions for obtaining the molecular electronic ground state, such as the boson-analogous construction recently proposed by Prof Dmitry Shalashilin (estimated duration: 6 months).
	\item Applying the fermionic $SU(M)$ methods for dynamics to other particle-preserving fermionic systems (e.g. qubit systems, electron gas with high chemical potential etc) (estimated duration: 6 months).
	\item Investigating the underlying connection between bosonic and fermionic $SU(M)$ coherent states and generalising it to other dynamical groups, which massively enlargens the scale of the proposed methods in terms of fermionic physical systems which may be analysed (estimated duration: 1 year in parallel with other work)
	\end{itemize}
	
	\section{Data management plan}
	The project is purely theoretical/computational, and thus no collection of laboratory data is required.
	
	Collection of numerical data from the frameworks developed by me has so far been done on my personal computer and stored on a github repository. As I move ahead to upscale the data collection, especially for large-mode bosonic systems and complex molecules, I will create a designated workflow for collecting data on the MacBook machine provided to me by the Faculty, so that long-term data collection may be put into place without interrupting the development workflow.
	
	In case the computational complexity of the largest systems we choose to analyse exceeds what may reasonably covered by a single machine, parallelisation or employing high-performance computing systems will be considered; however, given that the utility we are trying to demonstrate is that of a low-time-complexity approach, this will probably not be necessary.
	
	
	
	%\clearpage
	\begin{thebibliography}{10}

	\addcontentsline{toc}{chapter}{Bibliography}
	
	\bibitem{curse_of_dimensionality}
	Shalashilin, D. V. (2011), Multiconfigurational Ehrenfest approach to quantum coherent dynamics in large molecular systems. \textit{Faraday Discussions}, \textbf{153}, pp. 105--116
	
	\bibitem{semiclassical_generic_1}
	Crespo-Otero, R., Barbatti, M. (2018), Recent Advances and Perspectives on Nonadiabatic Mixed Quantum–Classical Dynamics. \textit{Chem. Rev.}, \textbf{118}, pp. 7026--7068
	
	\bibitem{semiclassical_generic_2}
	Mattos, R. S., Mukherjee, S., Barbatti, M. (2024), Quantum Dynamics from Classical Trajectories. \textit{J. Chem. Theory Comput.}, \textbf{20}, pp. 7728--7743
	
	\bibitem{mulliken}
	Mulliken, R. S. (1932), Electronic Structures of Polyatomic Molecules and Valence. II. General Considerations. \textit{Phys. Rev.}, \textbf{14}, pp. 49--71
	
	\bibitem{harmonic_classical_states}
	Schrodinger, E. (1926), Der stetige übergang von der Mikro-zur Makromechanik. \textit{Naturwiss}. 14, pp. 664-666. Translated into English in Schrodinger, E. (1928), \textit{Collected Papers in Wave Mechanics}. 1st edn. London: Blackie \& Son, pp. 41--44
	
	\bibitem{field_coherent_states}
	Glauber, R. J. (1963), Coherent and Incoherent States of the Radiation Field. \textit{Phys. Rev.}, \textbf{131}, 6, pp. 2766--2788
	
	\bibitem{no_unity}
	Klauder, J. R., Skagerstam, B. S. (1985), \textit{Coherent States: Applications in Physics and Mathematical Physics}. 1st eng. edn. Singapore: World Scientific.
	
	\bibitem{perelomov_og}
	Perelomov, A. M. (1972), Coherent states for arbitrary Lie group. \textit{Commun. Math. Phys.}, \textbf{26}, pp. 222--236
	
	\bibitem{gilmore_og}
	Gilmore, R. (1972), Geometry of symmetrized states. \textit{Ann. Phys.}, \textbf{74}, 2, pp. 391--463
	
	\bibitem{ZFG}
	Zhang, W. M., Feng, D. H., Gilmore, R. (1990), Coherent states: Theory and some applications. \textit{Rev. Mod. Phys.}, \textbf{62}, pp. 867--927
	
	\bibitem{aguiar}
	Viscondi, T. F., Grigolo, A., de Aguiar, M. A. M. (2015), Semiclassical Propagator in the Generalized Coherent-State Representation. \href{https://doi.org/10.48550/arXiv.1510.05952}{arXiv:1510.05952 \textbf{[quant-ph]}}
	
	\bibitem{optical_lattices}
	Block, I., Dalibard, J., Zwerger, W. (2008), Many-body physics with ultracold gases. \textit{Rev. Mod. Phys.}, \textbf{80}, pp. 885--964
	
	\bibitem{grossmann}
	Qiao, Y., Grossmann, F. (2021), Exact variational dynamics of the multimode Bose-Hubbard model based on $SU(M)$ coherent states. \textit{Phys. Rev. A}, \textbf{103}, 042209
	
	\bibitem{green}
	Green, J. A., Shalashilin, D. V. (2019), Simulation of the quantum dynamics of indistinguishable bosons with the method of coupled coherent states. \textit{Phys. Rev. A}, \textbf{100}, 013607
	
	\bibitem{gellmann}
	Bertlmann, R. A., Krammer, P. (2008), Bloch vectors for qudits. \href{ 	
https://doi.org/10.48550/arXiv.0806.1174}{arXiv:0806.1174 \textbf{[quant-ph]}}

	\bibitem{buonsante}
	Buonsante, P., Penna, V.(2008), Some remarks on the coherent-state variational approach to nonlinear boson models. \textit{J. Phys. A: Math. Theor.}, \textbf{41}, 175301
	
	\bibitem{sampling_algorithm}
	Grigolo, A., Viscondi, T. F., de Aguiar, M. A. M. (2016), Multiconfigurational quantum propagation with trajectory-guided generalized coherent states. \textit{J. Chem. Phys.}, \textbf{144}, 094106
	
	\bibitem{imaginary_timeprop}
	Pederiva, F., Roggero, A., Schmidt, K. E. (2017), "Variational and Diffusion Monte Carlo Approaches to the Nuclear Few- and Many-Body Problem" in \textit{An Advanced Course in Computational Nuclear Physics: Bridging the Scales from Quarks to Neutron Stars.} Springer International Publishing, Cham, pp. 401--476
	
	\bibitem{charpol}
	Horn, R. A., Johnson, C. R. (2012), \textit{Matrix Analysis}. 2nd edn. Cambridge University Press, Cambridge. \href{https://doi.org/10.1017/CBO9781139020411}{doi.org/10.1017/CBO9781139020411}
	
	\bibitem{modified_cauchy_binet}
	Chapman, A., Miyake, A. (2018), Classical simulation of quantum circuits by dynamical localization: analytic results for Pauli-observable scrambling in time-dependent disorder. Available at \href{https://arxiv.org/pdf/1704.04405v2}{arxiv.org/pdf/1704.04405v2 [quant-ph]}

	\end{thebibliography}
	\bibliographystyle{unsrt}
	
	\appendix
	
	\chapter{Auxilary theorems}
This appendix presents mathematical results which were used to derive the mathematical properties of bosonic and fermionic coherent states, but which were chosen to be omitted from the main body of text due to their abstract mathematical nature.

The author does not claim originality of these theorems or their proofs, but chooses to include them, as he was unable to find them in existing literature. The work presented in this appendix is the author's, except for mathematical identities which are stated explicitly.

\section{Lemmas for counting matrix minors under constraints}
In the following section, $\seq{x}$ for integer $x$ denotes the sequence $\seq{1,2\dots x}$, and $\seq{u}\oplus\seq{v}$ denotes the increasing sequence constructed from all the elements present in the sequences $\seq{u}, \seq{v}$.

\begin{lemma} \label{lemma:sum of constrained principal minors}
For a square matrix $M$, the sum of all principal minors of all ranks which contain the first $n$ rows/columns is equal to $\det(I^{(n)}+M)$, where $I^{(n)}$ is the identity matrix with the first $n$ elements along the diagonal set to zero.
\end{lemma}
\textit{Proof.} We will proceed by induction. Firstly, for $n=0$, the sum is the sum of all principal minors for each rank $r$, denoted $E_r$. These sums form the coefficients of the characteristic polynomial of $M$ like so: \cite[Th. 1.2.16]{charpol}
	\begin{equation}
	\det(tI-M)=\sum_{r=0}^x(-1)^{r} t^{x-r} E_r
	\end{equation}
	where $x$ is the number of rows of $M$. Evaluating this sum at $t=-1$ yields
	\begin{align}
	\det(-I-M)&=\sum_{r=0}^x(-1)^{r} (-1)^{x-r} E_r\nonumber\\
	\det(I+M)&=\sum_{r=0}^x E_r
	\end{align}
	which shows the lemma holds for $n=0$. Now: assume the lemma holds for $n$. For $n+1$, we have the sum of all principal minors of $M$ which contain the first $n+1$ rows/columns. We identify this as equivalent to the sum of all principal minors of $M$ which contain the first $n$ rows/columns, from which we subtract the sum of all principal minors of $M$ which contain the first $n$ rows/columns and do \textit{not} contain the $(n+1)$-th row/column. I.e. we can write
	\begin{align}
	\sum_{r=n+1}^{x}\sum_{\seq{a}\in\Gamma_r\seq{x^{(n+1)}}}&\det(M_{\seq{n+1}\oplus\seq{a},\seq{n+1}\oplus\seq{a}})=\\
	&\sum_{r=n}^{x}\sum_{\seq{a}\in\Gamma_r\seq{x^{(n)}}}\left[\det(M_{\seq{n}\oplus\seq{a},\seq{n}\oplus\seq{a}})-\det(M'_{\seq{n}\oplus\seq{a},\seq{n}\oplus\seq{a}})\right] \nonumber
	\end{align}
	where $M'$ is obtained by setting all elements in the $(n+1)$-th row/column to zero. To the right side of the equation we apply the lemma, since it is assumed it holds for $n$:
	\begin{equation} \label{eq: big minor sum from small minor sums}
	\sum_{r=n+1}^{x}\sum_{\seq{a}\in\Gamma_r\seq{x^{(n+1)}}}\det(M_{\seq{n+1}\oplus\seq{a},\seq{n+1}\oplus\seq{a}})=\det(I^{(n)}+M)-\det(I^{(n)}+M')
	\end{equation}
	Now, taking the Laplace expansion of $\det(I^{(n)}+M)$ along the $(n+1)$-th row, the element in the $(n+1)$-th column is $1+M_{n+1,n+1}$ and its cofactor is $\det(I^{(n)}+M')$. Subtracting $1$ from this element and adding the cofactor to the full Laplace expansion preserves the determinant, revealing
	\begin{equation} \label{eq: laplace expansion of small minor sum}
	\det(I^{(n)}+M)=\det(I^{(n+1)}+M)+\det(I^{(n)}+M')
	\end{equation}
	Substituing Eq. \ref{eq: laplace expansion of small minor sum} into Eq. \ref{eq: big minor sum from small minor sums} yields
	\begin{equation}
	\sum_{r=n+1}^{x}\sum_{\seq{a}\in\Gamma_r\seq{x^{(n+1)}}}\det(M_{\seq{n+1}\oplus\seq{a},\seq{n+1}\oplus\seq{a}})=\det(I^{(n+1)}+M)
	\end{equation}
	which finishes the proof.
	
	\begin{lemma}\label{lemma:symmetrically constrained sum of complementary minors}
	Consider two matrices $X, Y$ of shapes $(m,n)$ and $(n,m)$, respectively. The sum
	\begin{equation}
	\sum_{r=0}^{\min(m,n)}\sum_{\seq{a}\in\Gamma_r\seq{m^{(u)}}}\sum_{\seq{b}\in\Gamma_r\seq{n^{(v)}}}\det(X_{\seq{u}\oplus\seq{a},\seq{v}\oplus\seq{b}})\det(Y_{\seq{v}\oplus\seq{b},\seq{u}\oplus\seq{a}})
	\end{equation}
	where $\seq{x^{(y)}}$ signifies the sequence $\seq{x}$ with the first $y$ elements omitted, is equal to
	\begin{equation}
	(-1)^v\det(I^{(u+v)}+\mqty(0 & Y_{\text{r.} v} \\ X_{\text{c.} v} & X^{(\text{c.} v)} Y^{(\text{r.} v)}))=(-1)^u\det(I^{(u+v)}+\mqty(0 & X_{\text{r.} u} \\ Y_{\text{c.} u} & Y^{(\text{c.} u)} X^{(\text{r.} u)}))
	\end{equation}
	where subscript $\text{r. }z,\text{c. }z$ specifies the rows or columns of a submatrix by inclusion of the index sequence $z$, and the superscript $(\text{r. }z),(\text{c. }z)$ specifies the rows or columns of a submatrix by omission of the index sequence $z$.
	\end{lemma}
	\textit{Proof.} By applying the modified Cauchy-Binet formula \cite[App. C]{modified_cauchy_binet} we can contract either of the two sequences $\seq{a},\seq{b}$. Contracting sequence $\seq{a}$ yields
	\begin{equation}
	=(-1)^v\sum_{r=0}^{\min(m,n)}\sum_{\seq{a}\in\Gamma_r\seq{(m+v)^{(u+v)}}}\det(M_{\seq{u+v}\oplus\seq{a},\seq{u+v}\oplus\seq{a}})\qq{where}M=\mqty(0 & Y_{\text{r.} v} \\ X_{\text{c.} v} & X^{(\text{c.} v)}Y^{(\text{r.} v)})
	\end{equation}
	Applying lemma \ref{lemma:sum of constrained principal minors} directly yields the first result in the theorem. Contracting sequence $\seq{b}$ first and then applying the lemma yields the second result in the theorem. The proof is thus finished.
	
	\begin{lemma}\label{lemma:asymmetrically constrained sum of complementary minors}
	Consider the generalisation of \ref{lemma:symmetrically constrained sum of complementary minors}, where the transpose of $X$ no longer has the same dimensions as $Y$, and where the constraints on row/column inclusion for the two minors in each term of the sum are asymmetrical:
	\begin{equation}
	S=\sum_{r}\sum_{\seq{a}\in\Gamma_{r_a}\seq{m}}\sum_{\seq{b}\in\Gamma_{r_b}\seq{n}}\det(X_{u_x\oplus\seq{a},v_x\oplus\seq{b}})\det(Y_{v_y\oplus\seq{b},u_y\oplus\seq{a}})
	\end{equation}
	where $u_x+v_y=v_x+u_y$ and $v_x\leq v_y$, $r_a$ and $r_b$ are taken such that the submatrices in each term are square, the sum over $r$ includes all possible square minors of $X,Y$ which satisfy the constraints, and where $D_{k\oplus\seq{l},m\oplus\seq{n}}$ is the smallest submatrix of $D$ which contains the upper left block of $D$ with shape $(k,m)$, as well as the submatrix of the lower right block of $D$ given by the row index sequence $\seq{l}$ and column index sequence $\seq{n}$. The sum evaluates to
	\begin{equation}
	S=(-1)^{v_y(1+v_y-v_x)}\det(I^{(u_x+v_y)}+\mqty(
		0_{v_y,v_x} & Y_{\text{r.} v_y}\\
		X_{\text{c.} v_x} & X^{(\text{c.} v_x)} Y^{(\text{r.} v_y)}
	))
	\end{equation}
	\end{lemma}
	\textit{Proof.} We start with a few observations. Firstly, the constraint $u_x+v_y=v_x+u_y$ is not arbitrary, and it is in fact necessary for the sum to be constructable, which can be seen from inspecting the dimensions of $X,Y$--this also makes the choice of $r_a,r_b$, and the summation limits of $r$ unique. Secondly, the notation $D_{k\oplus\seq{l},m\oplus\seq{n}}$ can be thought of as $D_{\seq{x},\seq{y}}$, where $\seq{x}$ is formed from indices $1\dots k$ concatenated with the sequence $\seq{l}$ where each element was increased by $k$, with construction of $\seq{y}$ being analogous. Thirdly, the fact only one of two possible contractions is presented follows from the minimal assumption $v_x\leq v_y$, which loses no generality. Should the opposite be true, relabelling $X,Y$ and $\seq{a},\seq{b}$ reduces the problem to its original form; should we instead specify $u_x\leq u_y$, relabelling $X,Y$ \textit{or} $\seq{a},\seq{b}$ once again reduces the problem to the statement above.
	
	Denote $\Delta v = v_y-v_x$. Consider the matrix $X'=I_{\Delta v}\oplus X$, i.e.
	\begin{equation}
	X'=\mqty(\dmat{I_{\Delta v}, X})\qq{hence}\det(X'_{u_x+\Delta v\oplus\seq{a},v_x+\Delta v\oplus\seq{b}})=\det(X_{u_x\oplus\seq{a},v_x\oplus\seq{b}})
	\end{equation}
	We can use the modified Cauchy-Binet formula to contract $\seq{b}$ like so:
	\begin{align}
	S=(-1)^{v_y}\sum_{r=0}^{\min(m,n)}\sum_{\seq{a}\in\Gamma_{r_a}\seq{m}}\det(M_{u_x+v_y\oplus\seq{a}, u_y+v_y\oplus\seq{a}})\qq{where}M=\mqty(
		0_{v_y,v_y} & Y_{\text{r.} v_y}\\
		X'_{\text{c.} v_y} & X'^{(\text{c.} v_y)} Y^{(\text{r.} v_y)}
	)
	\end{align}
	We now observe that
	\begin{equation}
	X'_{\text{c.} v_y} = \mqty(
		I_{\Delta v} & 0_{\Delta v, v_x}\\
		0_{v_x, \Delta v} & X_{\text{c.} v_x}
	)
	\end{equation}
	and since the first $\Delta v$ rows of $X'^{(\text{c.} v_y)}$ are zero, we also have
	\begin{equation}
	X'^{(\text{c.} v_y)} Y^{(\text{r.} v_y)} = \mqty(0_{\text{r.} \Delta v} \\ X^{(\text{c.} v_x)} Y^{(\text{r.} v_y)})
	\end{equation}
	Taking the Laplace expansion of $M$ for all $(\Delta v,\Delta v)$ minors along the first $\Delta v$ columns, there is only one non-zero contribution, which is $\det(I_{\Delta v})=1$. Considering its cofactor for any arbitrary constrained minor of $M$, we can rewrite the sum as
	\begin{equation}
	S = (-1)^{v_y(1+\Delta v)}\sum_{r=0}^{\min(m,n)}\sum_{\seq{a}\in\Gamma_{r_a}\seq{m}}\det(M'_{u_x+v_y\oplus\seq{a},v_x+u_y\oplus\seq{a}})\qq{where}M'=\mqty(
		0_{v_y,v_x} & Y_{\text{r.} v_y}\\
		X_{\text{c.} v_x} & X^{(\text{c.} v_x)}Y^{(\text{r.} v_y)}
	)
	\end{equation}
	Since $u_x+v_y=v_x+u_y$, this sum can be reducing by direct application of Lemma \ref{lemma:sum of constrained principal minors}, which finishes the proof.
	
	\textit{Note.} We shall now explicitly state the construction of $r_a, r_b$ and the summation limits on $r$. Since we demand the submatrices of $X, Y$ be square, we have
	\begin{eqnarray}
	u_x + r_a = v_x + r_b = r \geq 0\\
	u_y + r_a = v_y + r_b = r \geq 0
	\end{eqnarray}
	To satisfy both equations and inequalities, we take
	\begin{multline}
	r_a=r - \min(u_x,u_y),\qquad r_b=r - \min(v_x,v_y),\\
	r = \max(\min(u_x, u_y), \min(v_x, v_y))\dots \min(X_{\text{rows}}-u_x+\min(u_x, u_y),X_{\text{cols}}-v_x+\min(v_x, v_y))
	\end{multline}
	
	
	
	
	

	
	
\end{document}