\documentclass[12pt]{article}
\usepackage[a4paper, total={6.5in, 8in}]{geometry}
\usepackage{xargs}
\usepackage{amsmath,amssymb}
\usepackage{hyperref}
\usepackage{physics}
\usepackage{graphicx}
\newcommand{\sgn}{\text{sgn}}
\newcommand{\seq}[1]{\langle #1\rangle}
\newcommand{\asc}[1]{\upharpoonleft #1 \upharpoonright}
\newcommand{\hc}{^\dagger}
\newcommand{\Sym}{\text{Sym}}

\newtheorem{full_coupling_hilbert_space}
{Theorem}

\begin{document}

	\title{Construction of $\Sym(M)$ fermionic coherent states on the particle-preserving dynamical group.}
	\author{Michal Horanský}	
	\maketitle
	
	\abstract{The dynamical group of a fermionic system with $M$ modes which preserves total particle number is identified as $G=\Sym(M)$. A reference state $\ket{\phi_0}$ is constructed as a member of the full occupancy basis by partitioning the modes into $\pi_1$ ($S$ occupied modes) and $\pi_0$ ($M-S$ unoccupied modes). The quotient space of $(G,\ket{\phi_0})$ is shown to be generated by $\hat{f}_i\hc\hat{f}_j$, where $i\in\pi_0,j\in\pi_1$, and a generalised coherent state $\ket{Z}$ belonging to this quotient space is decomposed into the full occupancy basis. The overlap element $\braket{Z_a}{Z_b}$ is expressed as a sum of the coefficients of the characteristic polynomial of $Z_a\hc Z_b$ with non-trivially alternating signs. The action of the "transposition operator" $\hat{f}_i\hc\hat{f}_j$ (for arbitrary $i,j$) on $\ket{Z}$ is expressed in the full occupancy basis, and the expression for a general two-body-interacting total-particle-preserving Hamiltonian matrix element $\mel{Z_a}{\hat{H}}{Z_b}$ is given. The time complexity of calculating said quantities is discussed.}
	
	\section{Construction of $\Sym(M)$ coherent states}
	
	\subsection{General approach to decomposition into the full occupancy basis}
	
	\subsection{Dynamical group, reference state, and quotient space}
	
	\subsection{$\Sym(M)$ decomposition into the full occupancy basis}
	
	
	\section{Properties of $\Sym(M)$ coherent states}
	
	\subsection{Overlap of two coherent states}
	
	\subsection{Action of the transposition operator}
	
	The transposition operator $\hat{T}_{ij}=\hat{f}\hc_i\hat{f}_j$ not only generates the quotient space of $(\Sym(M), \ket{\phi_0})$ when restricting the domain of $i,j$, but, in general, constitues any $S$-preserving operator. This can be readily seen from the fact that an arbitrary sequence $\hat{f}\hc_{a_1}\dots\hat{f}\hc_{a_X}\hat{f}_{b_1}\dots\hat{f}_{b_Y}$ can commute with $\hat{N}=\sum_{m=1}^M\hat{f}\hc_m\hat{f}_m$ only if $X=Y$, i.e. the numbers of creation and annihilation operators are equal. Since we have
	\begin{align}\label{eq:transposition commutes with N}
	\comm{\hat{T}_{ij}}{\hat{N}}&=\sum_{m=1}^M\comm{\hat{f}\hc_i\hat{f}_j}{\hat{f}\hc_m\hat{f}_m}\nonumber\\
	&=\sum_{m=1}^M\left(\comm{\hat{f}\hc_i}{\hat{f}\hc_m}\hat{f}_j\hat{f}_m+\hat{f}\hc_m\comm{\hat{f}\hc_i}{\hat{f}_m}\hat{f}_j+\hat{f}\hc_i\comm{\hat{f}_j}{\hat{f}\hc_m}\hat{f}_m+\hat{f}\hc_m\hat{f}\hc_i\comm{\hat{f}_j}{\hat{f}_m}\right)\nonumber\\
	&=\sum_{m=1}^M\left(2\hat{f}\hc_i\hat{f}\hc_m\hat{f}_j\hat{f}_m+\hat{f}\hc_m(2\hat{f}\hc_i\hat{f}_m-\delta_{im})\hat{f}_j+\hat{f}\hc_i(\delta_{jm}-2\hat{f}\hc_m\hat{f}_j)\hat{f}_m+2\hat{f}\hc_m\hat{f}\hc_i\hat{f}_j\hat{f}_m\right)\nonumber\\
	&=\sum_{m=1}^M\left(\hat{f}\hc_i\hat{f}_m\delta_{jm}-\hat{f}\hc_m\hat{f}_j\delta_{im}\right)=\hat{f}\hc_i\hat{f}_j-\hat{f}\hc_i\hat{f}_j=0
	\end{align}
	we see that any operator which commutes with $\hat{N}$ (other than the identity) can be expressed as a sum of products of $\hat{T}_{ij}$. Therefore, the action of $\hat{T}_{ij}$ on a coherent state $\ket{Z}$ is of great interest to us.
	
	\subsection{Matrix element of the quadratic $S$-preserving Hamiltonian}
	For an $S$-preserving Hamiltonian, the one-body interaction can be expressed as $V^{(1)}_{\alpha,\beta}\hat{f}\hc_\alpha\hat{f}_\beta$, and two-body interaction as $\frac{1}{2}V^{(2)}_{\alpha,\beta,\gamma,\delta}\hat{f}\hc_\alpha\hat{f}\hc_\beta\hat{f}_\gamma\hat{f}_\delta$, where
	\begin{itemize}
	\item $V^{(1)}$ is Hermitian
	\item $V^{(2)}$ is anti-symmetric w.r.t. exchange of the first or second pair of indices, and Hermitian w.r.t. exchange of the two pairs of indices.
	\end{itemize}
	Then
	\begin{equation}
	\hat{H}=V^{(1)}_{\alpha,\beta}\hat{f}\hc_\alpha\hat{f}_\beta+\frac{1}{2}V^{(2)}_{\alpha,\beta,\gamma,\delta}\hat{f}\hc_\alpha\hat{f}\hc_\beta\hat{f}_\gamma\hat{f}_\delta
	\end{equation}
	
	\appendix
	
	\section{Notation in this article}
	\begin{itemize}
		\item $\seq{S}$: A sequence constructed from the elements of set $S\subset \mathbb{N}^+$ such that\\ $\seq{S}_i< \seq{S}_j\iff i<j$. Such sequence shall be referred to as ascending. If $S$ is a number, the sequence is explicitly $\seq{1, 2\dots S}$.
		\item $\Gamma_n\seq{S}$: Set of all subsequences of length $n$ of sequence $\seq{S}$.
		\item $\seq{S_1}\oplus\seq{S_2}$: An ascending sequence constructed from ascending sequences $\seq{S_1}$ and $\seq{S_2}$ with no common elements, such that it contains every element from $\seq{S_1}$ and $\seq{S_2}$. The length of $\seq{S}$ shall be denoted as $\abs{\seq{S}}$.
		\item $M_{\seq{S_1},\seq{S_2}}$ where $M$ is a matrix: This denotes a matrix $M'$ such that $M'_{ij}=\nolinebreak M_{\seq{S_1}_i,\seq{S_2}_j}$, which is a submatrix of $M$.
		\item $\ket{\seq{S}}$: An element of the full occupancy basis where the $i$-th mode is occupied iff $i\in\seq{S}$.
		\item $\hat{f}\hc_{\seq{S}}$: A product of $N=\abs{\seq{S}}$ fermionic creation operators $\hat{f}\hc_{\seq{S}_1}\dots \hat{f}\hc_{\seq{S}_N}$. An analogous construction can be defined for a sequence of annihilation operators.
		\item $P^k$: The set of permutations of $k$ elements. For an element $P\in P^k$ and an ascending sequence $\seq{S}$, we denote $P\seq{S}_i$ the $i$-the element of the (not necessarily ascending) sequence constructed by permuting $\seq{S}$ by $P$.
		\item $\asc{\hat{f}\hc_{\sigma_1}\dots\hat{f}\hc_{\sigma_n}}$: The \textit{monotonic ordering} of a product of fermionic creation (or annihilation) operators. The result is the product of the same set of operators $\hat{f}\hc_{\rho_1}\dots\hat{f}\hc_{\rho_n}\equiv \hat{f}\hc_{\seq{\rho}}, \{\rho\}=\{\sigma\}$ such that their indices are in an ascending order. Equivalently, for any permutation $P\in P^n$, we have $\asc{\hat{f}\hc_{P\seq{\rho}}}=\hat{f}\hc_{\seq{\rho}}$. \textit{Note:} If applied to a sequence of both creation and annihilation operators, the monotonic ordering first applies a normal ordering, and then is applied to the creation and annihilation operators separately.
	\end{itemize}
	
	\section{Properties of fermionic creation and annihilation operators}
	
	Commutators and such
	
	\section{Invalidity of the boson-analogous construction}
	
	The $SU(M)$ bosonic coherent state with $S$ particles can be expressed as
	\begin{equation}
	\ket{z}=N(z)\left(\sum_{m=1}^Mz_m\hat{b}\hc_m\right)^S\ket{\text{vac.}}
	\end{equation}
	where $N(z)$ is some real-valued normalisation function. Let us create a "naive" fermionic coherent state with $S$ particles by replacing the bosonic creation operators by their fermionic counterparts:
	\begin{equation}
	\ket{z}=N(z)\left(\sum_{m=1}^Mz_m\hat{f}\hc_m\right)^S\ket{\text{vac.}}
	\end{equation}
	Expanding the multinomial product, we see that all terms with repeated creation operators $\hat{f}\hc_i\hat{f}\hc_i$ vanish, yielding
	\begin{equation*}
	\ket{z}=N(z)\sum_{\seq{a}\in\Gamma^S\seq{M}} \left(\prod_{i=1}^S z_{\seq{a}_i}\right) \sum_{P\in P^S} \hat{f}\hc_{P\seq{a}}=N(z)\sum_{\seq{a}\in\Gamma^S\seq{M}} \left(\prod_{i=1}^S z_{\seq{a}_i}\right) \hat{f}\hc_{\seq{a}}\sum_{P\in P^S}\sgn(P)
	\end{equation*}
	However, since $\sgn(P)$ is an irreducible representation of the permutation group on $P^S$, it is orthogonal to the trivial representation (for $S>1$), and hence its sum over all group elements vanishes. Hence
	\begin{enumerate}
		\item For $S=0,1$, the naive construction is equivalent to the construction in this article up to a meaningless transformation of the $z$ parameter.
		\item For $S>1$, the naive construction vanishes.
	\end{enumerate}
	
	
	\begin{thebibliography}{10}



\end{thebibliography}	
	
\end{document}