\documentclass[english]{beamer}
\mode<presentation>

%\usepackage[slovak]{babel}
\usepackage[utf8]{inputenc}
\usepackage{physics}
\usepackage{tikz, pgfplots}
%\DeclareMathSizes{12pt}{10pt}{10pt}{10pt}

\usepackage{multimedia}
\RequirePackage{environ}

\usepackage{changepage}
\makeatletter
\newenvironment{forcecenter}%
    {\@parboxrestore%
     \begin{adjustwidth}{}{\leftmargin}%
    }{\end{adjustwidth}
     }
\makeatother

%\graphicspath{images}

%Use Okular!

\newcommand\Wider[2][3em]{
\makebox{\linewidth][c]{
	\begin{minipage}{\dimexpr\textwidth+#1\relax}
	\raggedright#2
	\end{minipage}	
	}
}
}

\NewEnviron{sea}{
	\fontsize{9}{12}
    \begin{eqnarray*}
   % \fontsize{12pt}{10pt}\selectfont
    \BODY
    \end{eqnarray*}
    \normalsize
}

\begin{document}

  \title{Bosonic and fermionic coherent states\\\hfill\\A progress report}
  \author[Michal Horanský]{Michal Horanský\\\hfill\\University of Leeds}
  %\date{14. apríl 2019}
  %\maketitle
  \date{June 20, 2025}
  \begin{frame}
    \titlepage
  \end{frame}
  
  
  \begin{frame}
  	\frametitle{Bosonic $SU(M)$ CSs}
  	\framesubtitle{Progress on simulation of dynamics}
  	\begin{itemize}
  		\item Numerical issues solved; the data for $M=2,3$ is now sensible\\
  		\item Comparison with fully variational approach shows that the decoupled basis method reproduces the dynamics of a low-mode system sufficiently well--and saves a lot of time
  		\item So far, the initial wavefunction was a "pure" coherent state, which will be pefectly reproduced by a basis of size $1$ in the coherent timespan
  		\item What to do next:
  		\begin{itemize}
  			\item For arbitrary intitial wavefunctions, a sane sampling method must be implemented
  			\item Stability must be shown for large ($20-30$) number of modes
  			\item Minimal required basis size for these cases is to be found
  		\end{itemize}
  	\end{itemize}
  \end{frame}
  
  
  \begin{frame}
  	\frametitle{Fermionic $SU(M)$ CSs}
  	\framesubtitle{Construction}
  	\begin{itemize}
  		\item I found a construction of fermionic coherent states which
  		\begin{enumerate}
  			\item is wholly contained in the particle-preserving Hilbert space
  			\item doesn't use Grassmann algebra
  		\end{enumerate}
  		\item In this construction:
  		\begin{itemize}
  			\item Each CS is parametrised by $S(M-S)$ complex parameters
  			\item The overlap $\braket{Z_a}{Z_b}$ is calculable as a determinant in $O(M^3)$
  			\item The Hamiltonian matrix element $\mel{Z_a}{\hat{H}}{Z_b}$ is calculable as an $M^4$-term sum of determinant-like reduced overlaps, yielding a time complexity of $O(M^7)$.
  		\end{itemize}
  	\end{itemize}
  \end{frame}
  
  
  \begin{frame}
  	\frametitle{Fermionic $SU(M)$ CSs}
  	\framesubtitle{Utility}
  	\begin{itemize}
  		\item Can be used to calculate the dynamics of a molecular electronic Hamiltonian with complexity polynomial in $M$
  		\item This can be used to find the molecular ground state
  		\item Example methods:
  		\begin{itemize}
  			\item Krylov-Lanczos method
  			\item Imaginary time propagation
  		\end{itemize}
  	\end{itemize}
  \end{frame}
  
  \begin{frame}
  	\frametitle{Outlook}
  	\begin{itemize}
  		\item I am currently implementing a python framework to perform krylov-lanczos calculations on arbitrary molecules using the fermionic CS construction
  		\item Other things which would be useful:
  		\begin{itemize}
  			\item Can the Hamiltonain matrix element be calculated with smaller time complexity if optimised well?
  			\item Can a simple expression for the inverse of the Kähler potential of the fermionic $SU(M)$ CSs be found? (This would be equivalent to lowering the time complexity on single-CS dynamics calculations)
  			\item Formalism comparison with Dima
  		\end{itemize}
  	\end{itemize}
  \end{frame}
  
  \begin{frame}[plain,c]
  	\begin{center}
  	  \Huge Thank you for your attention
  	\end{center}
  \end{frame}


\end{document}